\section{Preprocessing Data (Tiền xử lý dữ liệu)}

\subsection{Giới thiệu chung về vai trò của tiền xử lý dữ liệu trong khai thác dữ liệu}

\hspace{0.5cm}Trong dự án này, tiền xử lý dữ liệu đóng vai trò quan trọng trong việc chuẩn bị dữ liệu về chất lượng không khí cho mô hình học máy. Dữ liệu thô (\textit{raw data}) chứa thông tin về các chỉ số ô nhiễm không khí, nhiệt độ, độ ẩm và thời gian, cần được xử lý để đảm bảo tính chính xác và hiệu quả của mô hình dự đoán.

\begin{figure}[htbp]
    \centering
    \includegraphics[width=0.9\textwidth]{images/raw_data_preview.png}
    \vspace{0.5cm}
    \caption{Hiển thị dữ liệu gốc trước khi tiền xử lý}
    \label{fig:raw_data_preview}
\end{figure}

\begin{figure}[htbp]
    \centering
    \includegraphics[width=0.45\textwidth]{images/raw_data_statistics.png}
    \vspace{0.5cm}
    \caption{Thống kê mô tả dữ liệu gốc}
    \label{fig:raw_data_statistics}
\end{figure}

\subsection{Ý nghĩa của các đặc trưng}

\begin{description}
    \item[CO(GT)]: Nồng độ Carbon Monoxide trong không khí, được tính toán là trung bình theo giờ (hourly averaged). Dữ liệu này được thu thập từ một máy phân tích tham chiếu (reference analyzer), có độ chính xác cao. Nồng độ CO được đo bằng mg/m³ (miligram trên mét khối). Carbon monoxide (CO) là một khí độc, không mùi và không màu, có thể gây nguy hiểm nếu nồng độ trong không khí quá cao. Khi nồng độ CO vượt quá mức an toàn, nó có thể gây ra các triệu chứng như chóng mặt, đau đầu, và thậm chí là ngộ độc, nguy hiểm đến tính mạng nếu nồng độ quá cao.

    \item[PT08.S1(CO)]: Phản hồi cảm biến theo giờ (hourly averaged sensor response) của một cảm biến được thiết kế để đo nồng độ khí carbon monoxide (CO). Dữ liệu này là giá trị đo được từ một cảm biến cụ thể (có thể là cảm biến điện hóa hoặc cảm biến quang học) được sử dụng để đo nồng độ CO. Cảm biến này không đo trực tiếp nồng độ CO bằng đơn vị mg/m³ hoặc ppm, mà thay vào đó là đo phản hồi của cảm biến (có thể là tín hiệu điện hay tín hiệu quang học).

    \item[NMHC(GT)]: Nồng độ trung bình theo giờ của các hợp chất hữu cơ không bão hòa (Non-Methane Hydrocarbons - NMHC) trong không khí. Dữ liệu này được thu thập từ một máy phân tích tham chiếu (reference analyzer), và giá trị trong cột này là nồng độ của các hợp chất này, đo bằng microgram mỗi mét khối (µg/m³). NMHC là một nhóm các hợp chất hữu cơ dễ bay hơi (VOCs) không chứa metan. Những hợp chất này thường xuất hiện trong khí thải từ các nguồn như giao thông, công nghiệp, hoặc từ các sản phẩm tiêu dùng như sơn và dung môi.

    \item[C6H6(GT)]: Nồng độ trung bình theo giờ của Benzen (C6H6) trong không khí, đo bằng microgram mỗi mét khối (µg/m³). Benzen là một hợp chất rất nguy hiểm nếu tiếp xúc lâu dài hoặc với nồng độ cao. Nó có thể gây ra các vấn đề nghiêm trọng về sức khỏe, bao gồm các bệnh về máu, ung thư và các vấn đề liên quan đến hệ thần kinh.

    \item[PT08.S2(NMHC)]: Phản hồi cảm biến theo giờ của một cảm biến được thiết kế để đo nồng độ các hợp chất hữu cơ không bão hòa (Non-Methane Hydrocarbons - NMHC) trong không khí. Dữ liệu này không đo nồng độ NMHC trực tiếp bằng các đơn vị như µg/m³ hay ppm, mà là giá trị phản hồi của cảm biến.

    \item[NOx(GT)]: Nồng độ trung bình theo giờ của Nitrogen Oxides (NOx) trong không khí, được đo bằng một máy phân tích tham chiếu. Đơn vị đo trong cột này là ppb (parts per billion). NOx là một nhóm các hợp chất bao gồm Nitrogen Dioxide (NO2) và Nitric Oxide (NO), có thể gây ảnh hưởng đến sức khỏe con người và môi trường.

    \item[PT08.S3(NOx)]: Phản hồi cảm biến theo giờ của một cảm biến được thiết kế để đo nồng độ Nitrogen Oxides (NOx) trong không khí. Dữ liệu này là phản hồi của cảm biến, tính trung bình trong suốt một giờ.

    \item[NO2(GT)]: Nồng độ trung bình theo giờ của Nitrogen Dioxide (NO2) trong không khí, được đo bằng một máy phân tích tham chiếu. Đơn vị đo của cột này là microgram per cubic meter (µg/m³). NO2 là một chất ô nhiễm quan trọng trong không khí, có thể gây ra các vấn đề sức khỏe nghiêm trọng.

    \item[PT08.S4(NO2)]: Phản hồi cảm biến theo giờ của một cảm biến được thiết kế để đo nồng độ Nitrogen Dioxide (NO2) trong không khí. Dữ liệu này là phản hồi cảm biến, tính trung bình trong suốt một giờ.

    \item[PT08.S5(O3)]: Phản hồi cảm biến theo giờ của một cảm biến được thiết kế để đo nồng độ Ozone (O3) trong không khí. Dữ liệu này là phản hồi của cảm biến, được tính trung bình trong suốt một giờ.

    \item[T]: Nhiệt độ (Temperature) trong không khí, được đo bằng đơn vị độ Celsius (°C). Nhiệt độ đóng vai trò quan trọng trong việc xác định điều kiện môi trường và ảnh hưởng đến sự phân tán và pha loãng các chất ô nhiễm.

    \item[RH]: Độ ẩm tương đối (Relative Humidity), được đo bằng đơn vị % (phần trăm). Đây là tỷ lệ giữa lượng hơi nước hiện có trong không khí và lượng hơi nước tối đa mà không khí có thể chứa tại một nhiệt độ nhất định. Độ ẩm tương đối ảnh hưởng lớn đến chất lượng không khí và các phản ứng hóa học trong khí quyển.

    \item[AH]: Độ ẩm tuyệt đối (Absolute Humidity), được đo bằng đơn vị g/m³ (gram trên mét khối). Đây là lượng hơi nước có trong không khí, không phụ thuộc vào nhiệt độ. Độ ẩm tuyệt đối là một chỉ số quan trọng để hiểu rõ hơn về sự thay đổi của lượng hơi nước trong không khí và ảnh hưởng của nó đến chất lượng không khí.

    \item[Date\_Time]: Thời gian đo đạc
    \item[Year, Month, Day, Hour]: Các đặc trưng thời gian
\end{description}

\subsection{Quá trình tiền xử lý dữ liệu}

\subsubsection{Bước 1: Xử lý các cột không cần thiết}
\hspace{0.5cm}Đầu tiên, nhóm loại bỏ các cột không cần thiết như "Unnamed: 15" và "Unnamed: 16" để làm sạch dữ liệu.

\subsubsection{Bước 2: Xử lý giá trị thiếu}
\hspace{0.5cm}Đối với các cột số học, giá trị thiếu được thay thế bằng giá trị trung bình của cột đó. Đối với các cột dạng chuỗi, giá trị thiếu được thay thế bằng giá trị xuất hiện thường xuyên nhất.

\subsubsection{Bước 3: Chuyển đổi kiểu dữ liệu}
\hspace{0.5cm}Các cột dữ liệu được chuyển đổi sang kiểu dữ liệu phù hợp:
\begin{itemize}
    \item Cột "Date" được chuyển đổi từ định dạng DD/MM/YYYY sang datetime
    \item Cột "Time" được chuyển đổi từ định dạng HH.MM.SS sang time
    \item Các cột số học được chuẩn hóa bằng cách thay thế dấu phẩy bằng dấu chấm
    \item Cột CO(GT) có giá trị -200 được thay thế bằng giá trị trung bình
\end{itemize}

\subsubsection{Bước 4: Xử lý ngoại lệ (outliers)}
\hspace{0.5cm}Nhóm sử dụng phương pháp Z-score để xác định và xử lý các giá trị ngoại lệ:
\begin{itemize}
    \item Tính toán Z-score cho mỗi cột số học
    \item Xác định ngưỡng là 3 độ lệch chuẩn
    \item Thay thế các giá trị ngoại lệ bằng giá trị trung bình của cột
    \item Đặc biệt xử lý cột NMHC(GT) để đảm bảo không có giá trị âm
\end{itemize}

\begin{figure}[htbp]
    \centering
    \includegraphics[width=0.9\textwidth]{images/data_distribution_comparison.png}
    \vspace{0.5cm}
    \caption{So sánh phân phối dữ liệu trước và sau khi xử lý ngoại lệ}
    \label{fig:data_distribution}
\end{figure}

\subsubsection{Bước 5: Tạo đặc trưng mới}
\hspace{0.5cm}Nhóm tạo các đặc trưng mới từ dữ liệu hiện có:
\begin{itemize}
    \item Kết hợp cột "Date" và "Time" thành cột "Date\_Time"
    \item Trích xuất thông tin năm, tháng, ngày và giờ từ cột "Date\_Time"
    \item Tạo các đặc trưng tương tác như Temp\_Humidity, NOx\_NO2
    \item Thêm các đặc trưng thống kê như rolling mean và rolling std
\end{itemize}

\subsection{Kết quả tiền xử lý dữ liệu}

\subsubsection{Thông tin tổng quan về dữ liệu}
\hspace{0.5cm}Sau khi tiền xử lý, bộ dữ liệu có các đặc điểm sau:
\begin{itemize}
    \item Số lượng mẫu: 9,471
    \item Số lượng đặc trưng: 18
    \item Các kiểu dữ liệu:
    \begin{itemize}
        \item float64: 13 cột
        \item datetime64[ns]: 1 cột
        \item int32: 4 cột
    \end{itemize}
\end{itemize}

\begin{figure}[htbp]
    \centering
    \includegraphics[width=0.9\textwidth]{images/processed_data_preview.png}
    \vspace{0.5cm}
    \caption{Hiển thị dữ liệu sau khi tiền xử lý}
    \label{fig:processed_data_preview}
\end{figure}

\subsubsection{Thống kê mô tả}
\hspace{0.5cm}Các thống kê chính của dữ liệu sau tiền xử lý:
\begin{itemize}
    \item CO(GT): Giá trị trung bình 0.77, dao động từ -5.12 đến 11.90
    \item PT08.S1(CO): Giá trị trung bình 1097.15, dao động từ 647 đến 2008
    \item NMHC(GT): Giá trị trung bình 213.58, dao động từ 29 đến 405
    \item C6H6(GT): Giá trị trung bình 10.06, dao động từ 0.10 đến 63.70
\end{itemize}

\begin{figure}[htbp]
    \centering
    \includegraphics[width=0.45\textwidth]{images/processed_data_statistics.png}
    \vspace{0.5cm}
    \caption{Thống kê mô tả dữ liệu sau tiền xử lý}
    \label{fig:processed_data_statistics}
\end{figure}

\subsection{Kết luận}
\hspace{0.5cm}Quá trình tiền xử lý dữ liệu đã giúp chuẩn bị bộ dữ liệu chất lượng không khí cho việc xây dựng mô hình học máy. Các bước xử lý đã giải quyết các vấn đề về giá trị thiếu, ngoại lệ và tạo thêm các đặc trưng có ý nghĩa. Kết quả cho thấy dữ liệu đã được chuẩn hóa và sẵn sàng cho việc huấn luyện mô hình.