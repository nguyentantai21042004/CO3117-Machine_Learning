\section{Mô hình Linear Regression}
\subsection{Cài đặt và huấn luyện mô hình}

\subsubsection{Tiền xử lý dữ liệu}
\begin{itemize}
    \item \textbf{Đọc dữ liệu:} Từ file CSV phân cách bằng dấu `;`, chuẩn hoá dữ liệu kiểu châu Âu (dấu phẩy `,` chuyển thành chấm `.`).
    \item \textbf{Xử lý dữ liệu thiếu:} Giá trị \texttt{-200} được xem là thiếu và thay bằng \texttt{NaN}. Các cột toàn \texttt{NaN} hoặc không hợp lệ (\texttt{Unnamed}) bị loại bỏ.
    \item \textbf{Biến thời gian:} Kết hợp \texttt{Date} và \texttt{Time} thành \texttt{DateTime}, sau đó trích xuất thêm các đặc trưng \texttt{Hour}, \texttt{Day}, \texttt{Month}, \texttt{DayOfWeek}.
\end{itemize}

\subsubsection{Xử lý đặc trưng}
\begin{itemize}
    \item Loại bỏ các cột không phải kiểu số hoặc không còn cần thiết.
    \item Loại bỏ các hàng có \texttt{NaN} trong biến mục tiêu \texttt{CO(GT)}.
    \item Sử dụng \texttt{KNNImputer} để xử lý giá trị thiếu trong tập đặc trưng.
\end{itemize}

\subsubsection{Kỹ thuật mở rộng đặc trưng (Feature Engineering)}
\begin{itemize}
    \item Tạo thêm 3 đặc trưng phi tuyến:
    \begin{itemize}
        \item \texttt{PT08.S1(CO)} $\times$ \texttt{NOx(GT)}
        \item \texttt{PT08.S5(O3)}$^2$
        \item \texttt{NO2(GT)} $\times$ \texttt{RH}
    \end{itemize}
\end{itemize}

\subsubsection{Xử lý ngoại lệ (Outliers)}
\begin{itemize}
    \item Sử dụng phương pháp IQR để loại bỏ outliers ở cả tập đặc trưng và mục tiêu.
\end{itemize}

\subsubsection{Chia dữ liệu}
\begin{itemize}
    \item Chia dữ liệu thành tập huấn luyện (80\%) và tập kiểm thử (20\%).
    \item Chuẩn hóa dữ liệu bằng \texttt{StandardScaler}.
\end{itemize}

\subsubsection{Huấn luyện mô hình}
\begin{itemize}
    \item Sử dụng mô hình \textbf{Linear Regression với ràng buộc hệ số không âm} (\texttt{positive=True}) để đảm bảo các đặc trưng không có tác động âm tới \texttt{CO(GT)}.
    \item Mô hình được huấn luyện trên tập huấn luyện đã chuẩn hóa.
\end{itemize}

\subsection{Tuning Hyperparameters}

\subsubsection{Đặc điểm}
\begin{itemize}
    \item Linear Regression không có quá nhiều siêu tham số. Trong code, có sử dụng:
    \begin{itemize}
        \item \texttt{positive=True}: Ép các hệ số của mô hình phải lớn hơn hoặc bằng 0, phù hợp với các đặc trưng vật lý như nồng độ khí, độ ẩm, v.v.
    \end{itemize}
\end{itemize}

\subsubsection{Chiến lược tối ưu hoá gián tiếp}
\begin{itemize}
    \item Tối ưu chất lượng dữ liệu đầu vào thông qua:
    \begin{itemize}
        \item Xử lý giá trị thiếu bằng \texttt{KNNImputer}.
        \item Loại bỏ outliers với IQR.
        \item Thêm đặc trưng phi tuyến $\Rightarrow$ giúp mô hình tuyến tính mô tả tốt hơn mối quan hệ phi tuyến.
    \end{itemize}
\end{itemize}

\noindent \textit{\textbf{Lưu ý:}} Không sử dụng \texttt{GridSearchCV} hay \texttt{Cross-Validation}, do mô hình đơn giản và quá trình tối ưu tập trung vào xử lý dữ liệu đầu vào.

\subsection{Kết quả và đánh giá}

\subsubsection{Đánh giá trên tập kiểm thử}
\begin{itemize}
    \item \textbf{Mean Squared Error (MSE):} 0.0907
    \item \textbf{Root MSE (RMSE):} 0.3012
    \item \textbf{Mean Absolute Error (MAE):} 0.2106
    \item \textbf{R\textsuperscript{2} score:} 0.8704 $\Rightarrow$ mô hình giải thích khoảng 87\% phương sai của dữ liệu.
\end{itemize}

\subsubsection{Diễn giải hệ số}
\begin{itemize}
    \item Các hệ số được trích xuất và sắp xếp theo mức độ ảnh hưởng.
    \item \textbf{Top 5 đặc trưng ảnh hưởng mạnh nhất (theo trị tuyệt đối hệ số):}
\end{itemize}

\begin{center}
\begin{tabular}{|l|c|}
\hline
\textbf{Feature} & \textbf{Coefficient} \\
\hline
C6H6(GT) & 0.542707 \\
PT08.S1(CO) & 0.145489 \\
NOx(GT) & 0.131271 \\
PT08.S3(NOx) & 0.129269 \\
NO2(GT) & 0.075051 \\
\hline
\end{tabular}
\end{center}

\subsubsection{Biểu đồ trực quan}
\begin{itemize}
    \item So sánh giá trị dự đoán và thực tế.
    \item Phân phối và phân tán sai số.
    \item Mức độ quan trọng của các đặc trưng.
\end{itemize}

\subsubsection{Dự đoán thử và kiểm tra sai số}
\begin{itemize}
    \item Thử dự đoán 5 dòng từ tập kiểm thử và in sai số từng dòng.
    \item Sai số trung bình: \textbf{0.1556}, cho thấy độ chệch không quá lớn.
\end{itemize}
