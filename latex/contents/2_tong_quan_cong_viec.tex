\section{Tổng quan công việc}

\subsection{Giới thiệu về phương pháp luận}

\hspace{0.5cm}Trong dự án này, chúng tôi áp dụng phương pháp luận CRISP-DM (Cross-Industry Standard Process for Data Mining) để thực hiện quá trình khai thác dữ liệu. Phương pháp này bao gồm 6 giai đoạn chính:

\begin{itemize}
    \item \textbf{Hiểu biết về nghiệp vụ (Business Understanding):} Xác định mục tiêu và yêu cầu của dự án
    \item \textbf{Hiểu biết về dữ liệu (Data Understanding):} Thu thập và phân tích dữ liệu ban đầu
    \item \textbf{Chuẩn bị dữ liệu (Data Preparation):} Tiền xử lý và làm sạch dữ liệu
    \item \textbf{Mô hình hóa (Modeling):} Xây dựng và đánh giá các mô hình
    \item \textbf{Đánh giá (Evaluation):} Đánh giá kết quả và kiểm tra mục tiêu
    \item \textbf{Triển khai (Deployment):} Triển khai và duy trì mô hình
\end{itemize}

\subsection{Phương pháp tiếp cận}

\hspace{0.5cm}Dự án sử dụng phương pháp tiếp cận dựa trên dữ liệu (Data-Driven Approach) để phân tích và dự đoán năng suất cây trồng. Các bước thực hiện bao gồm:

\begin{itemize}
    \item \textbf{Thu thập dữ liệu:}
    \begin{itemize}
        \item Sử dụng các nguồn dữ liệu đa dạng (CSV, Excel)
        \item Tích hợp dữ liệu từ nhiều nguồn khác nhau
        \item Đảm bảo tính nhất quán và độ tin cậy của dữ liệu
    \end{itemize}
    
    \item \textbf{Tiền xử lý dữ liệu:}
    \begin{itemize}
        \item Xử lý dữ liệu thiếu và trùng lặp
        \item Chuẩn hóa và mã hóa dữ liệu
        \item Phát hiện và xử lý ngoại lệ
    \end{itemize}
    
    \item \textbf{Phân tích dữ liệu:}
    \begin{itemize}
        \item Phân tích thống kê mô tả
        \item Phân tích tương quan giữa các biến
        \item Phân tích xu hướng theo thời gian
    \end{itemize}
    
    \item \textbf{Xây dựng mô hình:}
    \begin{itemize}
        \item Lựa chọn các thuật toán phù hợp
        \item Huấn luyện và tối ưu hóa mô hình
        \item Đánh giá hiệu suất mô hình
    \end{itemize}
\end{itemize}

\subsection{Các kỹ thuật sử dụng}

\hspace{0.5cm}Dự án áp dụng các kỹ thuật khai thác dữ liệu sau:

\begin{itemize}
    \item \textbf{Phân tích thống kê:}
    \begin{itemize}
        \item Thống kê mô tả (mean, median, mode, standard deviation)
        \item Phân tích tương quan (correlation analysis)
        \item Kiểm định giả thuyết (hypothesis testing)
    \end{itemize}
    
    \item \textbf{Học máy:}
    \begin{itemize}
        \item Học có giám sát (supervised learning)
        \item Học không giám sát (unsupervised learning)
        \item Học bán giám sát (semi-supervised learning)
    \end{itemize}
    
    \item \textbf{Xử lý dữ liệu:}
    \begin{itemize}
        \item Chuẩn hóa dữ liệu (MinMaxScaler)
        \item Mã hóa one-hot (one-hot encoding)
        \item Xử lý ngoại lệ (outlier detection)
    \end{itemize}
    
    \item \textbf{Trực quan hóa dữ liệu:}
    \begin{itemize}
        \item Biểu đồ phân tán (scatter plots)
        \item Biểu đồ đường (line plots)
        \item Biểu đồ boxplot
        \item Heatmap tương quan
    \end{itemize}
\end{itemize}

\subsection{Đánh giá và kiểm chứng}

\hspace{0.5cm}Quá trình đánh giá và kiểm chứng được thực hiện thông qua:

\begin{itemize}
    \item \textbf{Phân chia dữ liệu:}
    \begin{itemize}
        \item Tập huấn luyện (training set)
        \item Tập kiểm định (validation set)
        \item Tập kiểm tra (test set)
    \end{itemize}
    
    \item \textbf{Đánh giá mô hình:}
    \begin{itemize}
        \item Độ chính xác (accuracy)
        \item Precision và Recall
        \item F1-score
        \item RMSE (Root Mean Square Error)
        \item MAE (Mean Absolute Error)
    \end{itemize}
    
    \item \textbf{Kiểm chứng chéo:}
    \begin{itemize}
        \item K-fold cross validation
        \item Leave-one-out cross validation
    \end{itemize}
\end{itemize}

\subsection{Tính mới và đóng góp}

\hspace{0.5cm}Dự án mang lại các đóng góp mới trong lĩnh vực khai thác dữ liệu nông nghiệp:

\begin{itemize}
    \item \textbf{Phương pháp tích hợp dữ liệu:}
    \begin{itemize}
        \item Tích hợp dữ liệu từ nhiều nguồn khác nhau
        \item Xử lý tự động các vấn đề về dữ liệu
        \item Đảm bảo tính nhất quán của dữ liệu
    \end{itemize}
    
    \item \textbf{Cải tiến trong xử lý dữ liệu:}
    \begin{itemize}
        \item Tự động phát hiện và xử lý ngoại lệ
        \item Chuẩn hóa dữ liệu thông minh
        \item Mã hóa dữ liệu phân loại hiệu quả
    \end{itemize}
    
    \item \textbf{Ứng dụng thực tế:}
    \begin{itemize}
        \item Dự đoán năng suất cây trồng
        \item Phân tích ảnh hưởng của các yếu tố môi trường
        \item Hỗ trợ ra quyết định trong nông nghiệp
    \end{itemize}
\end{itemize}

\subsection*{Tóm Tắt}
Phương pháp luận của dự án kết hợp các kỹ thuật khai thác dữ liệu hiện đại với quy trình CRISP-DM, tập trung vào việc xử lý và phân tích dữ liệu nông nghiệp. Các kỹ thuật được áp dụng bao gồm phân tích thống kê, học máy, và trực quan hóa dữ liệu, nhằm đạt được kết quả chính xác và có ý nghĩa thực tiễn. 