\section{Giới thiệu}

\subsection{Bối cảnh nghiên cứu}
\hspace{0.5cm}Ô nhiễm không khí là một trong những vấn đề môi trường nghiêm trọng nhất hiện nay, đặc biệt là tại các khu vực đô thị. Trong đó, Carbon Monoxide (CO) và Carbon Dioxide (CO2) là hai trong số các chất ô nhiễm không khí quan trọng cần được theo dõi và kiểm soát. CO là một khí độc không màu, không mùi, có thể gây nguy hiểm đến sức khỏe con người khi tiếp xúc ở nồng độ cao. CO2, mặc dù ít độc hơn, nhưng là một trong những khí nhà kính chính góp phần vào biến đổi khí hậu toàn cầu.

\subsection{Mục tiêu nghiên cứu}
\hspace{0.5cm}Dự án này tập trung vào việc phát triển các mô hình học máy để dự đoán nồng độ CO và CO2 trong không khí dựa trên bộ dữ liệu Air Quality từ UCI Machine Learning Repository. Cụ thể, các mục tiêu chính bao gồm:

\begin{itemize}
    \item Phân tích và xử lý dữ liệu từ các cảm biến chất lượng không khí
    \item Xây dựng các mô hình dự đoán nồng độ CO và CO2
    \item Đánh giá hiệu suất của các mô hình dự đoán
    \item Phân tích tầm quan trọng của các yếu tố ảnh hưởng đến nồng độ CO và CO2
\end{itemize}

\subsection{Bộ dữ liệu}
\hspace{0.5cm}Bộ dữ liệu Air Quality được thu thập từ một thiết bị đa cảm biến khí được triển khai tại một thành phố ở Ý. Dữ liệu bao gồm:

\begin{itemize}
    \item Thời gian thu thập: Từ tháng 3/2004 đến tháng 2/2005
    \item Số lượng mẫu: 9,358 mẫu
    \item Tần suất: Trung bình theo giờ
    \item Các đặc trưng chính:
    \begin{itemize}
        \item Nồng độ CO (mg/m³)
        \item Nồng độ các hợp chất hữu cơ không chứa metan (µg/m³)
        \item Nồng độ Benzen (µg/m³)
        \item Nồng độ NOx (ppb)
        \item Nồng độ NO2 (µg/m³)
        \item Nhiệt độ (°C)
        \item Độ ẩm tương đối (\%)
        \item Độ ẩm tuyệt đối (g/m³)
    \end{itemize}
\end{itemize}

\subsection{Ý nghĩa thực tiễn}
\hspace{0.5cm}Nghiên cứu này có ý nghĩa quan trọng trong việc:

\begin{itemize}
    \item Cảnh báo sớm về mức độ ô nhiễm không khí
    \item Hỗ trợ các cơ quan quản lý môi trường trong việc ra quyết định
    \item Nâng cao nhận thức cộng đồng về chất lượng không khí
    \item Góp phần vào việc phát triển các hệ thống giám sát chất lượng không khí thông minh
\end{itemize}

\subsection{Cấu trúc báo cáo}
\hspace{0.5cm}Báo cáo được tổ chức thành các phần chính sau:

\begin{itemize}
    \item Phần 1: Giới thiệu tổng quan về đề tài
    \item Phần 2: Cơ sở lý thuyết về học máy và các mô hình dự đoán
    \item Phần 3: Phương pháp nghiên cứu và xử lý dữ liệu
    \item Phần 4: Kết quả thực nghiệm và đánh giá
    \item Phần 5: Kết luận và hướng phát triển
\end{itemize}
